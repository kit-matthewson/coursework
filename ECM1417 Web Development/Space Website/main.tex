\documentclass{article}

\usepackage[a4paper,margin={2cm,2.5cm}]{geometry}
\usepackage[T1]{fontenc}
\usepackage{graphicx}
\usepackage{booktabs}
\usepackage{float}
\usepackage{parskip}

\title{\textbf{ECM1417: Web Development} \\ \textit{Coursework 1}}
\author{730002704}
\date{Tuesday, 18th February 2025}

\begin{document}

\maketitle
\tableofcontents

\section*{Declaration}
AI-supported use is permitted in this assessment. I acknowledge the following uses of GenAI tools in this assessment:
\begin{itemize}
    \item I have used GenAI tools to assist with research or gathering information.
    \item I have used GenAI tool to generate images, figures or diagrams.
\end{itemize}
I declare that I have referenced use of GenAI outputs within my assessment in line with the University referencing guidelines.

\section{Web Design}
\subsection{Purpose and Target Audience}
The target audience for the website is people interested in space exploration, focusing on amateurs. It aims to educate and engage visitors by covering significant milestones and future activities.

The content is simple with a lot of images so that it is easy for anyone to understand. The website is designed to be visually appealing and easy to navigate.

\subsection{Content Selection, Organization, and Design}
The website has four pages: Home, History, Future, and Contact. The Home page provides an overview of the website, the History page covers significant milestones in space exploration, the Future page discusses future activities, and the Contact page provides contact information.

NASA's official website was useful for gathering information about space exploration. I found twelve significant milestones for the history page, and then found an image to accompany each one.

The home page features a hero image (generated by AI) that is meant to inspire visitors.

Asking GenAI for key milestones in space exploration helped me to select the content for the history page. The NASA website was also useful for finding information about future activities.

The website uses a consistent blue colour scheme. This is used for the navigation bar, headings, buttons, and to highlight images. Arial is used across the whole website.

\subsection{Key Technologies}
The website uses HTML and CSS to meet the specification requirements. Some CSS animations are used to enhance user interactions.

I used a global stylesheet so that the website is consistent across all pages. This stylesheet contains the colour scheme, font, and other common styles.

The timeline for the future page required a large amount of unique CSS, and so I included this in the head of the HTML file for that page.

\subsection{Challenges and Solutions}
The hardest part of the development was the timeline for the future page. This required learning about relative and absolute positioning, and using a pseudo element to create the vertical line.

Creating circular images for the future page required some research into CSS properties. I found that setting the border radius to 50\% would create a circle. Transforming the images smoothly on hover was achieved using the \texttt{transition} property with an ease timing function.

Another challenge was embedding a Google Map in the contact page. Some research revealed that an \texttt{iframe} could be used to embed the map without needing an API key.

\section{Development Log}
\begin{table}[H]
    \centering
    \begin{tabular}{lll}
        \toprule
        \textbf{Data} & \textbf{Time} & \textbf{Duration} \\
        \midrule
        27/01/2025    & 14:00         & 1 hour            \\
        29/01/2025    & 15:00         & 2 hours           \\
        02/02/2025    & 12:00         & 3 hours           \\
        03/02/2025    & 16:00         & 1 hour            \\
        05/02/2025    & 14:00         & 2 hours           \\
        06/02/2025    & 15:00         & 1 hour            \\
        09/02/2025    & 12:00         & 3 hours           \\
        10/02/2025    & 16:00         & 1 hour            \\
        12/02/2025    & 14:00         & 2 hours           \\
        13/02/2025    & 15:00         & 1.5 hours         \\
        \bottomrule
    \end{tabular}
\end{table}
\end{document}
